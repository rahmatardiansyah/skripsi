\chapter{PENDAHULUAN}
\pagenumbering{arabic}

\section{Latar Belakang}
Pada era digital saat ini, pengolahan citra digital menjadi salah satu bidang yang semakin berkembang dan penting dalam dunia teknologi informasi dan komunikasi. Pengolahan citra digital digunakan dalam berbagai aplikasi, seperti dalam industri kesehatan, keamanan, multimedia, dan lain sebagainya.

Namun dalam proses pembelajaran pengolahan citra digital, mahasiswa sering mengalami kesulitan dalam memahami konsep-konsep yang abstrak dan kompleks. Selain itu, media pembelajaran yang digunakan belum banyak memanfaatkan teknologi terkini, sehingga kurang interaktif dan tidak memadai untuk mendukung proses pembelajaran.

Oleh karena itu, penulis memilih topik ``\textbf{Media Pembelajaran Pengolahan Citra Digital Menggunakan Visualisasi Interaktif Berbasis Web}'' sebagai judul proposal tugas akhir. Tujuan dari penelitian ini adalah untuk mengembangkan media pembelajaran yang lebih interaktif dan mudah dipahami dengan memanfaatkan teknologi web dan visualisasi interaktif.

Penelitian ini dilakukan karena belum banyak media pembelajaran pengolahan citra digital yang memanfaatkan teknologi web dan visualisasi interaktif. Penelitian sebelumnya juga belum memanfaatkan teknologi web secara maksimal dalam pembelajaran pengolahan citra digital. Oleh karena itu, penelitian ini penting untuk mengisi kekosongan tersebut dan memberikan alternatif media pembelajaran yang lebih inovatif dan efektif dalam mendukung proses pembelajaran pengolahan citra digital.

Metodologi yang dipilih dalam penelitian ini adalah pengembangan media pembelajaran berbasis web dengan menggunakan model pengembangan ADDIE (Analysis, Design, Development, Implementation, dan Evaluation) yang terdiri dari lima tahap. Penelitian ini akan dilaksanakan dengan melakukan uji coba terhadap media pembelajaran yang dikembangkan kepada mahasiswa yang sedang mempelajari mata kuliah pengolahan citra digital.

Manfaat yang diharapkan dari penelitian ini adalah terciptanya media pembelajaran yang lebih interaktif dan mudah dipahami oleh mahasiswa. Selain itu, media pembelajaran ini diharapkan dapat membantu meningkatkan kualitas pembelajaran pengolahan citra digital dan mempermudah mahasiswa dalam memahami konsep-konsep yang kompleks dan abstrak. Dengan demikian, penelitian ini dapat memberikan sumbangsih yang positif bagi dunia pendidikan dan pengembangan teknologi informasi dan komunikasi.

\section{Identifikasi Masalah}
Berikut adalah beberapa identifikasi masalah yang dapat diambil dari latar belakang di atas:

\begin{enumerate}
  \item Kurangnya media pembelajaran yang interaktif dan mudah dipahami oleh mahasiswa dalam proses pembelajaran pengolahan citra digital.
  \item Mahasiswa sering mengalami kesulitan dalam memahami konsep-konsep yang abstrak dan kompleks pada pembelajaran pengolahan citra digital.
  \item Kurangnya penggunaan teknologi web dalam pembelajaran pengolahan citra digital, padahal teknologi ini dapat memberikan fitur-fitur interaktif yang lebih baik daripada media pembelajaran tradisional.
  \item Tidak banyaknya media pembelajaran pengolahan citra digital yang memanfaatkan visualisasi interaktif dalam pembelajarannya, sehingga sulit untuk memvisualisasikan konsep-konsep yang abstrak dan kompleks dalam pengolahan citra digital.
\end{enumerate}

\section{Rumusan Masalah}
Berdasarkan latar belakang di atas, maka dapat dirumuskan masalah utama dari proposal tugas akhir ini sebagai berikut:

Bagaimana mengembangkan media pembelajaran pengolahan citra digital yang lebih interaktif dan mudah dipahami oleh mahasiswa dengan memanfaatkan teknologi web dan visualisasi interaktif?

Untuk menjawab masalah tersebut, diperlukan beberapa pertanyaan penelitian yang lebih spesifik, yaitu:

\begin{enumerate}
    \item Apa saja konsep-konsep dasar yang harus dipahami oleh mahasiswa dalam pembelajaran pengolahan citra digital?
    \item Bagaimana mengembangkan media pembelajaran pengolahan citra digital yang interaktif dan mudah dipahami dengan memanfaatkan teknologi web dan visualisasi interaktif?
    \item Bagaimana mengukur efektivitas media pembelajaran yang telah dikembangkan?
    \item Bagaimana tingkat kepuasan mahasiswa terhadap penggunaan media pembelajaran yang dikembangkan?
\end{enumerate}

\section{Batasan Masalah}
Tujuan dari tugas akhir ini adalah untuk mengembangkan media pembelajaran pengolahan citra digital dengan menggunakan visualisasi interaktif berbasis web, dengan bahasa pemrograman JavaScript. Untuk mencapai tujuan tersebut, beberapa batasan masalah harus ditetapkan, yaitu:

\begin{enumerate}
    \item Penggunaan bahasa pemrograman: Tugas akhir ini akan menggunakan bahasa pemrograman JavaScript untuk mengembangkan media pembelajaran pengolahan citra digital.
    \item Lingkup pengolahan citra digital: Lingkup pengolahan citra digital yang akan digunakan dalam tugas akhir ini meliputi teknik pengolahan citra segmentasi.
    \item Pengembangan visualisasi interaktif: Tugas akhir ini akan mengembangkan media pembelajaran yang menggunakan visualisasi interaktif untuk mempermudah pemahaman konsep-konsep pengolahan citra digital.
    \item Pembuatan aplikasi berbasis web: Media pembelajaran akan dikembangkan dalam bentuk aplikasi berbasis web sehingga dapat diakses dengan mudah melalui berbagai perangkat.
    \item Fungsi dan fitur aplikasi: Aplikasi media pembelajaran yang dikembangkan akan memiliki beberapa fungsi dan fitur, seperti tampilan interaktif, kontrol dan manipulasi citra, dan pemrosesan citra secara \textit{real-time}.
    \item Pengujian dan evaluasi: Tugas akhir ini akan melakukan pengujian dan evaluasi terhadap media pembelajaran yang dikembangkan untuk memastikan efektivitas dan efisiensi dari aplikasi yang dibuat.
    \item Batasan waktu pengembangan: Pengembangan media pembelajaran pengolahan citra digital dengan visualisasi interaktif berbasis web ini akan dilakukan dalam waktu tertentu, sehingga batasan waktu pengembangan akan menjadi salah satu batasan yang harus diperhatikan dalam tugas akhir ini.
\end{enumerate}

\section{Tujuan Penelitian}
Tujuan dari penelitian tugas akhir ini adalah untuk mengembangkan media pembelajaran yang lebih interaktif dan mudah dipahami dengan memanfaatkan teknologi web dan visualisasi interaktif. Selain itu, penelitian ini juga bertujuan untuk meningkatkan kualitas pembelajaran pengolahan citra digital dan mempermudah mahasiswa dalam memahami konsep-konsep yang kompleks dan abstrak. Tujuan akhir dari penelitian ini adalah memberikan sumbangsih yang positif bagi dunia pendidikan dan pengembangan teknologi informasi dan komunikasi.

\section{Manfaat Penelitian}
Berikut adalah beberapa manfaat yang dapat diperoleh melalui penelitian tugas akhir ini:
\begin{enumerate}
    \item Meningkatkan kualitas pembelajaran pengolahan citra digital: Dengan adanya media pembelajaran yang lebih interaktif dan mudah dipahami, diharapkan kualitas pembelajaran pengolahan citra digital dapat meningkat.
    \item Membantu mahasiswa memahami konsep-konsep yang kompleks dan abstrak: Dalam pembelajaran pengolahan citra digital, mahasiswa sering mengalami kesulitan dalam memahami konsep-konsep yang abstrak dan kompleks. Dengan adanya media pembelajaran yang lebih inovatif dan efektif, diharapkan dapat membantu mahasiswa dalam memahami konsep-konsep tersebut.
    \item Memberikan alternatif media pembelajaran yang lebih modern: Media pembelajaran yang saat ini digunakan masih terbatas dan belum banyak memanfaatkan teknologi terkini. Dengan adanya media pembelajaran yang memanfaatkan teknologi web dan visualisasi interaktif, diharapkan dapat memberikan alternatif media pembelajaran yang lebih modern.
    \item Memberikan sumbangsih positif bagi dunia pendidikan dan pengembangan teknologi informasi dan komunikasi: Penelitian ini diharapkan dapat memberikan sumbangsih positif bagi dunia pendidikan dan pengembangan teknologi informasi dan komunikasi. Dengan adanya media pembelajaran yang lebih inovatif dan efektif, diharapkan dapat membantu meningkatkan kualitas pendidikan dan pengembangan teknologi informasi dan komunikasi.
\end{enumerate}
